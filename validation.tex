\documentclass{standalone}
\usepackage{tikz}
\tikzset{vector/.style={->, >=stealth, very thick, cap=round}}
\definecolor{col0}{HTML}{FFFFFF}
\definecolor{col1}{HTML}{FF7878}
\definecolor{col2}{HTML}{51B5F8}
\definecolor{col3}{HTML}{68E1AA}
\definecolor{col4}{HTML}{B869EA}
\definecolor{col5}{HTML}{FF5500}
\definecolor{col6}{HTML}{FF7878}
\definecolor{col7}{HTML}{FF7500}
\definecolor{col8}{HTML}{FF4F93}

\usepackage{pythontex}

\begin{document}
\tikzset{
	vector/.style={->, >=stealth, very thick, cap=round},
	dline/.style={dotted, cap=round},
	wall/.style={very thick},
}

%\begin{pycode}
%from math import inf
%from libpq import *
%
%t = inf
%while t == inf or t < 0.5 or t > 10:
%		p1 = Particle(
%						pos = np.zeros(2),
%						vel = np.random.uniform(-2, 2, 2),
%						radius = np.random.uniform(0.2, 2),
%						)
%		p2 = Particle(
%						pos = np.random.uniform(-20, 20, 2),
%						vel = np.random.uniform(-1, 1, 2),
%						radius = np.random.uniform(0.2, 2),
%						)
%		p1.mass = p1.radius
%		p2.mass = p2.radius
%
%		t = time_to_pp_collision(p1, p2)
%
%# Params before
%x1, y1 = p1.pos
%v1x, v1y = p1.vel
%m1, R1 = p1.mass, p1.radius
%x2, y2 = p2.pos
%v2x, v2y = p2.vel
%m2, R2 = p2.mass, p2.radius
%
%# Params after
%p1.move(t)
%p2.move(t)
%x1_, y1_ = p1.pos
%x2_, y2_ = p2.pos
%u1, u2 = pp_collision(p1, p2)
%v1_x, v1_y = u1
%v2_x, v2_y = u2
%
%drawstr = """
%\\begin{{tikzpicture}}
%	\\draw[dline] ({},{}) -- ({},{});
%	\\draw[dline] ({},{}) -- ({},{});
%	\\draw[draw=col1, fill=col1!30] ({},{}) circle ({});
%	\\draw[draw=col2, fill=col2!30] ({},{}) circle ({});
%	\\draw[vector] ({},{}) -- ++({},{});
%	\\draw[vector] ({},{}) -- ++({},{});
%	\\draw[draw=col1, fill=col1!30] ({},{}) circle ({});
%	\\draw[draw=col2, fill=col2!30] ({},{}) circle ({});
%	\\draw[vector] ({},{}) -- ++({},{});
%	\\draw[vector] ({},{}) -- ++({},{});
%\\end{{tikzpicture}}
%""".format(
%						x1, y1, x1_, y1_,
%						x2, y2, x2_, y2_,
%						x1, y1, R1,
%						x2, y2, R2,
%						x1, y1, v1x, v1y,
%						x2, y2, v2x, v2y,
%						x1_, y1_, R1,
%						x2_, y2_, R2,
%						x1_, y1_, v1_x, v1_y,
%						x2_, y2_, v2_x, v2_y,
%					)
%print(drawstr)
%\end{pycode}

\begin{pycode}
from math import inf
from libpq import *

t = inf
while t == inf or t < 1 or t > 20:
		p = Particle(
						pos = np.zeros(2),
						vel = np.random.uniform(-2, 2, 2),
						radius = np.random.uniform(0.5, 1),
						)
		w = Wall(
						center = np.random.uniform(-20,20,2),
						d1 = np.random.uniform(-10,10,2)
						)
		t = time_to_pw_collision(p, w)

wcx, wcy = w.center
wd1x, wd1y = w.d1
wnx, wny = w.normal
px, py = p.pos
pvx, pvy = p.vel
R = p.radius
x_s = p.pos + p.vel*(t + p.radius/np.linalg.norm(p.vel))
x_s_x, x_s_y = x_s

# Advance particle
p.move(t)
px_, py_ = p.pos
p.vel = pw_collision(p, w)
pvx_, pvy_ = p.vel

drawstr = """
\\begin{{tikzpicture}}
	\\draw[dline] ({},{}) -- ({},{});
	\\draw[wall] ({},{}) -- ++({},{});
	\\draw[wall] ({},{}) -- ++({},{});
	\\draw[vector] ({},{}) -- ++({},{});
	\\draw[draw=col1, fill=col1!30] ({},{}) circle ({});
	\\draw[vector] ({},{}) -- ++({},{});
	\\draw[draw=col1, fill=col1!30] ({},{}) circle ({});
	\\draw[vector] ({},{}) -- ++({},{});
	\\draw[fill=col1] ({},{}) circle (0.2);
\\end{{tikzpicture}}
""".format(
						px, py, px_, py_,
						wcx, wcy, wd1x, wd1y,
						wcx, wcy, -wd1x, -wd1y,
						wcx, wcy, wnx, wny,
						px, py, R,
						px, py, pvx, pvy,
						px_, py_, R,
						px_, py_, pvx_, pvy_,
						x_s_x, x_s_y,
					)
print(drawstr)
\end{pycode}
\end{document}
